\documentclass{article}
\pagestyle{empty}

% set up page size and margins
%\usepackage[letterpaper,landscape,centering,margin=0.5in]{geometry}
%\usepackage[screen,centering,margin=0.5in]{geometry}
\usepackage[paperwidth=233mm,paperheight=180mm,centering,margin=0.5in]{geometry}

% this was recommended by psnfss2e.pdf
%\usepackage{textcomp}
%\usepackage{txfonts}
%\renewcommand{\rmdefault}{cm}
%\usepackage{lmodern}
%\usepackage[Symbol]{upgreek}
%\newcommand{\alphaup}{\upalpha}
%%% Upright Greek letters
\usepackage{amsfonts}
\DeclareSymbolFont{UPM}{U}{eur}{m}{n}
\SetSymbolFont{UPM}{bold}{U}{eur}{b}{n}
\DeclareMathSymbol{\upalpha}{0}{UPM}{"0B}
\DeclareMathSymbol{\upbeta}{0}{UPM}{"0C}
%\usepackage[math]{kurier}
%\renewcommand{\rmdefault}{cm}
\usepackage[T1]{fontenc}
\usepackage{cmbright}
%\usepackage{ccfonts}

%\usepackage{psfonts}

%\usepackage{helvet}
%\usepackage{avant}

\usepackage{url}

%% % from http://www.superstrate.net/useful/useful.html
%% \DeclareFontFamily{U}{euc}{}% I chose euc because the chart is called Euler cursive
%% \DeclareFontShape{U}{euc}{m}{n}{<-6>eurm5<6-8>eurm7<8->eurm10}{}%
%% \DeclareSymbolFont{AMSc}{U}{euc}{m}{n} % I chose AMSc because AMSa and AMSb are defined in the amsfonts-package
%% \DeclareMathSymbol{\alphaup}{\mathord}{AMSc}{11}

% scale up the fonts and space things correctly


%\linespread{2}

%% stolen from file `a0poster.cls', modified by RHK shifted font sizes
%% down three lines (and added three smaller sizes)
\renewcommand{\tiny}        {\fontsize{6.95}{8.68}\selectfont}    
\renewcommand{\scriptsize}  {\fontsize{8.33}{10.42}\selectfont}    
\renewcommand{\footnotesize}{\fontsize{10}{12.5}\selectfont}    
\renewcommand{\small}       {\fontsize{12}{14}\selectfont}    
\renewcommand{\normalsize}  {\fontsize{14.4}{18}\selectfont}  
\renewcommand{\large}       {\fontsize{17.28}{22}\selectfont} 
\renewcommand{\Large}       {\fontsize{20.74}{25}\selectfont} 
\renewcommand{\LARGE}       {\fontsize{24.88}{30}\selectfont} 
\renewcommand{\huge}        {\fontsize{26.5}{32}\selectfont} 
\renewcommand{\Huge}        {\fontsize{35.83}{45}\selectfont} 
\newcommand{\veryHuge}      {\fontsize{43}{54}\selectfont}    
\newcommand{\VeryHuge}      {\fontsize{51.6}{64}\selectfont}  
\newcommand{\VERYHuge}      {\fontsize{61.92}{77}\selectfont} 
%			    {\fontsize{74.3}{93}\selectfont}  
%			    {\fontsize{89.16}{112}\selectfont}
%			    {\fontsize{107}{134}\selectfont}  

% These colours are tried and tested for titles and headers. Don't
% over use color!
\usepackage{color}
\definecolor{DarkBlue}{rgb}{0.1,0.1,0.5}
\definecolor{Red}{rgb}{0.8,0.0,0.0}
\definecolor{Purple}{rgb}{0.8,0.0,0.8}
\definecolor{LightRed}{rgb}{1.0,0.0,0.0}
\definecolor{Cyan}{rgb}{0.0,0.9,0.9}
\definecolor{Magenta}{rgb}{0.9,0.0,0.9}
\definecolor{Green}{rgb}{0.0,0.9,0.0}
\definecolor{PicBlue}{rgb}{0.098,0.082,0.227}

%% stolen from http://www.edwardtufte.com/tufte/
\definecolor{ETYellow}{rgb}{1,1,0.953}

%% I've made it darker to calm the effect when projected on a large
%% screen
\definecolor{BGYellow}{rgb}{0.95,0.95,0.85}
% \definecolor{BGYellow}{rgb}{0.9,0.9,0.7}
%\pagecolor{BGYellow}

% see documentation for a0poster class for the size options here
\def\Head#1{\noindent\begin{center}{\Large\color{DarkBlue}#1}\end{center}}
\def\Important#1{\noindent{\color{Red} #1}}
\def\ImportantB#1{\noindent{ #1}}
\def\LHead#1{\noindent{\Large #1}}
\def\Title#1{\noindent\begin{center}{\huge\color{Red}#1}\end{center}}
\def\Subtitle#1{\noindent\begin{center}{\LARGE\color{DarkBlue}#1}\end{center}}

% select the right font
%\renewcommand{\familydefault}{\sfdefault}

% The textpos package is necessary to position textblocks at arbitary 
% places on the page.
\usepackage[absolute,overlay]{textpos}

\usepackage{amsmath}
\usepackage{amssymb}

% Graphics to include graphics. Times is nice on posters, but you
% might want to switch it off and go for CMR fonts.
%\usepackage{graphics,wrapfig,times}
\usepackage{graphicx,wrapfig}

\TPGrid[0.5in,0.5in]{10}{10} % 2 cols of width 4.5, plus 1 gap of width 1

\parindent=0pt
\parskip=0.5\baselineskip

%\newcommand\ion[2]{#1{\scshape{#2}}} 
\newcommand\ion[2]{#1{\MakeUppercase{#2}}} 
\newcommand{\nH}{\ensuremath{n_\mathrm{H}}}
\newcommand{\Hn}{\ensuremath{\mathrm{HI}}}
\newcommand{\Hp}{\ensuremath{\mathrm{HII}}}
\newcommand{\Hen}{\ensuremath{\mathrm{HeI}}}
\newcommand{\Hep}{\ensuremath{\mathrm{HeII}}}
\newcommand{\Hepp}{\ensuremath{\mathrm{HeIII}}}
\newcommand{\logNH}{\ensuremath{\log(N_\mathrm{H}/\mathrm{cm}^{-2})}}

\begin{document}

\pagecolor{white}
\begin{textblock}{10}(0,1)

\Title{Quasars and \ion{H}{ii} regions during reionization}

\end{textblock}

\begin{textblock}{10}(0,4)

\Head{\large Roban Hultman Kramer \hfill with Zolt\'an Haiman}

\centering Department of Astronomy, Columbia University

\end{textblock}

$\;$

\vfill
{\small
\texttt{roban@astro.columbia.edu}\hfill\texttt{zoltan@astro.columbia.edu}}

%%%%%%%%%%%%%%%%%%%%%%%%%%%%%%%%%%%%%%%%%%%%%%%%%%%%%%%%%%%%%%%%%%%%%%%%%%%%%%
\
\clearpage
%%%%%%%%%%%%%%%%%%%%%%%%%%%%%%%%%%%%%%%%%%%%%%%%%%%%%%%%%%%%%%%%%%%%%%%%%%%%%%

\begin{textblock}{10}(0,0)
  \LHead{\color{Red}The thickness of high-redshift quasar ionization
    fronts}
\end{textblock}

\begin{textblock}{10}(0,0.5)
\Important{\color{DarkBlue}Could the thickness of the ionization front be observed and
  used to constrain the hardness of the quasar's ionizing spectrum?}

  \begin{center}
    Simulated propagation of quasar ionization fronts into the IGM
    \includegraphics[height=6\TPVertModule]{ionization_profile.png}\\ 
    {\color{Red}Ionized} and {\color{DarkBlue}neutral} fractions at
    $10^6$ and $10^8$ years for --- hard and - - soft spectra
  \end{center}

  With a sufficiently hard ionizing spectrum, the thickness of the
  front exceeds $\sim 1$ physical Mpc and may be measurable from the
  3D morphology of its redshifted 21-cm signal or Lyman-$\upalpha$
  absorption spectrum.

  {\small Kramer \& Haiman (2008 MNRAS v385 p1561)}
\end{textblock}

%%%%%%%%%%%%%%%%%%%%%%%%%%%%%%%%%%%%%%%%%%%%%%%%%%%%%%%%%%%%%%%%%%%%%%%%%%%%%%
\
\clearpage
%%%%%%%%%%%%%%%%%%%%%%%%%%%%%%%%%%%%%%%%%%%%%%%%%%%%%%%%%%%%%%%%%%%%%%%%%%%%%%
\pagecolor{PicBlue}
\begin{textblock}{10}(0,0)
  \LHead{\color{white} Probing reionization with quasar spectra}\\
%{The Impact of the Intrinsic Lyman-$\upalpha$ Emission Line Shape Uncertainty}
\end{textblock}

 \begin{textblock}{2.5}(7,3)
   \textcolor{white}{The damping wing of neutral hydrogen in the IGM
     extends into the transmission window of the quasar's ionized
     region.}
 \end{textblock}


\begin{textblock}{5}(0,9.2)
\textcolor{white}{\small
Djorgovski et al.\\ (2001 ApJ 560L 5)\\
\url{http://www.astro.caltech.edu/~george/reion/}}
\end{textblock}

 \begin{textblock}{5}(0,1)
   \includegraphics[height=9.5\TPVertModule]{discovexplbig_edit.jpg}
 \end{textblock}

%%%%%%%%%%%%%%%%%%%%%%%%%%%%%%%%%%%%%%%%%%%%%%%%%%%%%%%%%%%%%%%%%%%%%%%%%%%%%%
\
\clearpage
%%%%%%%%%%%%%%%%%%%%%%%%%%%%%%%%%%%%%%%%%%%%%%%%%%%%%%%%%%%%%%%%%%%%%%%%%%%%%%
\pagecolor{white}
\begin{textblock}{10}(0,0)
  \LHead{\color{Red} Probing reionization with quasar spectra:}\\
{\color{DarkBlue}  The Impact of the Intrinsic Lyman-$\upalpha$ Emission Line Shape Uncertainty}
\end{textblock}

\begin{textblock}{5}(0,1)
    \includegraphics[height=9.5\TPVertModule]{depth_figure.png}
\end{textblock}

\begin{textblock}{4.5}(5.5,1.5)
 The quasar absorption spectrum carries the signature of the IGM damping
 wing. Mesinger \& Haiman (2007) found best fit values:
\begin{center}
  {\large $x_\mathrm{IGM} = 0.2,~1.0,~1.0$}
\end{center}
at $z = 6.42, 6.28, 6.22$, suggesting that reionization is incomplete
at $z \sim 6.2$.

Calculating the optical depth from the observed absorption spectrum
requires modeling the intrinsic spectrum of the quasar.

Our main question:\\ 
\Important{Could errors in modeling and extrapolating the intrinsic
  flux bias the $x_\mathrm{IGM}$ value and cause an ionized IGM to
  mimic a neutral one?}
\end{textblock}

%%%%%%%%%%%%%%%%%%%%%%%%%%%%%%%%%%%%%%%%%%%%%%%%%%%%%%%%%%%%%%%%%%%%%%%%%%%%%%
\
\clearpage
%%%%%%%%%%%%%%%%%%%%%%%%%%%%%%%%%%%%%%%%%%%%%%%%%%%%%%%%%%%%%%%%%%%%%%%%%%%%%%

\begin{textblock}{10}(0,0)
\LHead{\color{Red} Recovering the IGM neutral fraction:}\\
input: $R_\mathrm{HII}=40.5\ \mathrm{Mpc}$, $x_\mathrm{Ref} = 10^{-5.5}$, $x_\mathrm{IGM} = 0.1$
\end{textblock}

\begin{textblock}{10}(0,1.25)
  \centering
  Fit results with \textbf{extrapolated spectra}:
  \includegraphics[width=7\TPHorizModule]{fig12b.png}

\raggedright
\Important{\Large Systemic errors in the intrinsic flux modeling bias the measurements
toward underestimating $x_\mathrm{IGM}$.}
\end{textblock}

%%%%%%%%%%%%%%%%%%%%%%%%%%%%%%%%%%%%%%%%%%%%%%%%%%%%%%%%%%%%%%%%%%%%%%%%%%%%%%
\
\clearpage
%%%%%%%%%%%%%%%%%%%%%%%%%%%%%%%%%%%%%%%%%%%%%%%%%%%%%%%%%%%%%%%%%%%%%%%%%%%%%%

\begin{textblock}{10}(0,0)
\LHead{\color{Red} Recovering the IGM neutral fraction:}\\
input: $R_\mathrm{HII}=40.5\ \mathrm{Mpc}$, $x_\mathrm{Ref} = 10^{-5.5}$, $x_\mathrm{IGM} = 0.1$
\end{textblock}

\begin{textblock}{10}(0,1)
\begin{center}
\includegraphics[width=5\TPHorizModule]{fig12b.png}
\end{center}

{\Large
\Important{ Systemic errors in the intrinsic flux modeling bias
  the measurements toward underestimating $x_\mathrm{IGM}$.}

This bias only strengths the conclusions in MH2007 that their best-fit
values (of $x_\mathrm{IGM} = 1.0$ at $z = 6.22$ and $6.28$) indicate a
significantly neutral IGM at $z>6$.}

\end{textblock}

%%%%%%%%%%%%%%%%%%%%%%%%%%%%%%%%%%%%%%%%%%%%%%%%%%%%%%%%%%%%%%%%%%%%%%%%%%%%%%
\
\clearpage
%%%%%%%%%%%%%%%%%%%%%%%%%%%%%%%%%%%%%%%%%%%%%%%%%%%%%%%%%%%%%%%%%%%%%%%%%%%%%%

\pagecolor{white}
\begin{textblock}{10}(0,0)
\LHead{\color{Red} Feedback from clustered sources during reionization}
\end{textblock}

\begin{textblock}{10}(0,0.5)
  \Important{\color{DarkBlue}How much will the clustering of early
    galaxies enhance their influence (feedback) on one another?}

  Feedback mechanism: star formation is suppressed in minihalos
  forming inside an ionized region.

  We find that clustering increases minihalo suppression by up to $\sim 60$
  times. This prolongs reionization and reduces $\tau$, as required by
  WMAP. Galaxy clustering is likely to similarly boost other feedback
  mechanisms.

\end{textblock}

\begin{textblock}{6.5}(0,3.1)
  \centering
    \Important{IGM ionized fraction versus redshift}\\
    \includegraphics[width=6.5\TPHorizModule]{feedback_curves.jpg}\\
\end{textblock}

\begin{textblock}{3}(7,3.1)
  \vspace{\baselineskip}

  {\small
  {\color{Purple} - - feedback with no clustering},\\
  --- feedback with clustering,\\
  {\color{DarkBlue} $\cdot \cdot$ all halos}, 
  {\color{Red} $\cdot \cdot$ large halos only}}

  \vspace{2\baselineskip}

  Feedback also increases the mean ionized bubble size at late stages
  of reionization.

  \vspace{2\baselineskip}

  {\small Kramer, Haiman, \& Oh\\ (2006 ApJ v649 p570)}

\end{textblock}

%%%%%%%%%%%%%%%%%%%%%%%%%%%%%%%%%%%%%%%%%%%%%%%%%%%%%%%%%%%%%%%%%%%%%%%%%%%%%%
\
\clearpage
%%%%%%%%%%%%%%%%%%%%%%%%%%%%%%%%%%%%%%%%%%%%%%%%%%%%%%%%%%%%%%%%%%%%%%%%%%%%%%

\begin{textblock}{10}(0,10)\centering
\textbf{\color{white} Honeycomb Cowfish \textit{Acanthostracion polygonius},
  Bonaire, Netherlands Antilles}
\end{textblock}

\begin{textblock}{12}(-1,-1)
    \includegraphics[width=12\TPHorizModule]{cowfish_1024.jpg}
\end{textblock}

%%%%%%%%%%%%%%%%%%%%%%%%%%%%%%%%%%%%%%%%%%%%%%%%%%%%%%%%%%%%%%%%%%%%%%%%%%%%%%
\
\clearpage
%%%%%%%%%%%%%%%%%%%%%%%%%%%%%%%%%%%%%%%%%%%%%%%%%%%%%%%%%%%%%%%%%%%%%%%%%%%%%%

\begin{textblock}{10}(0,0)
\LHead{\color{Red} Recovering the IGM neutral fraction:}\\
input: $R_\mathrm{HII}=40.5\ \mathrm{Mpc}$, $x_\mathrm{Ref} = 10^{-5.5}$, $x_\mathrm{IGM} = 0.1$
\end{textblock}

\begin{textblock}{10}(0,1)\centering
  \ImportantB{Histogram of fit results with {\color{DarkBlue}
      perfectly known spectra} (no flux extrapolation errors):}\\
  \includegraphics[width=8\TPHorizModule]{fig12a.png}
\end{textblock}

\end{document}
